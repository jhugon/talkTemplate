\documentclass{hugontalk}

\title{LArIAT Calibration}
%\subtitle{Updates}

\date{\today}
%\date{April 18, 2014}
%\date{Calibration Meeting \\ \today}

\author{Justin Hugon \\ Louisiana State University}
%\author{Justin Hugon \\ Louisiana State University \vspace{1em} \\ On Behalf of the Calibration Group}


%%%%%%%%%%%%%%%%%%%%%%%%%%%%%%%%%%

\begin{document}

\begin{frame}
  \maketitle
\vspace{-1em}
  \includegraphics[height=12mm]{/home/jhugon/Pictures/officialImages/Lariat_Logo.jpg}
  \hfill
  \includegraphics[height=16mm]{/home/jhugon/Pictures/officialImages/lsu/LSU_Full_Name_Purple_RGB.pdf}
\end{frame}

\makeatletter
\setbeamertemplate{footline}
{
    \begin{beamercolorbox}[colsep=1.5pt]{upper separation line foot}
    \end{beamercolorbox}
    \begin{beamercolorbox}[ht=2.5ex,dp=1.125ex,%
      leftskip=.3cm,rightskip=.3cm plus1fil]{author in head/foot}%
      \leavevmode{\usebeamerfont{author in head/foot}\insertshortauthor}%
      \hfill%
      {\usebeamerfont{institute in head/foot}\usebeamercolor[fg]{institute in head/foot}\insertshortinstitute}%
    \end{beamercolorbox}%
    \begin{beamercolorbox}[ht=2.5ex,dp=1.125ex,%
      leftskip=.3cm,rightskip=.3cm plus1fil]{title in head/foot}%
      {\usebeamerfont{title in head/foot}\insertshorttitle \hfill \insertframenumber}%
    \end{beamercolorbox}%
    \begin{beamercolorbox}[colsep=1.5pt]{lower separation line foot}
    \end{beamercolorbox}
}
\makeatother

\author{Justin Hugon, Louisiana State University}

%%%%%%%%%%%%%%%%%%%%%%%%%%%%%%%%%%
%%%%%%%%%%%%%%%%%%%%%%%%%%%%%%%%%%
%%%%%%%%%%%%%%%%%%%%%%%%%%%%%%%%%%
%%%%%%%%%%%%%%%%%%%%%%%%%%%%%%%%%%

\begin{frame}
\frametitle{Question from Reviewer}
%\scriptsize
"The choice of log-normal PDFs for the rate uncertainties - I believe
indeed this is a good choice, to avoid negative rates that would be
possible due to a Gaussian. I would like to ask whether a check on this
choice of PDF has been done. i.e. whether another positive definite PDF
with the same expectation value and variance for these rate
uncertainties has been checked, and the differences shown to be
negligible. While I don't expect problems for the log-normal nature for
the PDF, I believe this is a useful check to make."
\end{frame}

\begin{frame}
\frametitle{Statistics Committee}
\begin{itemize}
  \item I asked the StatCom for any advice on the Reviewer question
        \textcolor{blue}{\tiny\underline{\url{https://hypernews.cern.ch/HyperNews/CMS/get/statistics/421.html?outline=-1}}}
  \item Responses:
  \begin{itemize}
    \item Check the parameter pulls; if small, the PDF shape matters little
    \item Try log-Uniform distribution as extreme case
    \item Gamma distribution is a standard alternative to log-Normal
    \item Plot log-Normal \& Gamma distributions for your uncertainty values; check if they differ much
  \end{itemize}
\end{itemize}
\begin{center}
%\textbf{We Seek Final Approval from Higgs PAG}
\end{center}
\end{frame}

\begin{frame}
\frametitle{Pulls on Nuisance Parameters}
\begin{center}
        \begin{tikzpicture}
            \node[anchor=south west,inner sep=0] at (0,0) {
                \includegraphics[height=55mm]{plots/AllNuisances.pdf}
            };
            %\draw[help lines,xstep=1,ystep=1] (0,0) grid (8.0,8.0);
            \draw (4.8,5.4) node [fill=white,font=\tiny,text width=1.5cm,align=center] {$\mathrm{H\to\mu\mu}$ \\ Combination};
        \end{tikzpicture}
        \\
        \vspace{1ex}
        \bf Nuisance parameters aren't pulled
\end{center}
\end{frame}

\begin{frame}
\frametitle{Limits for 2-Jet VBF Tight 8 TeV: Gamma PDF}
\begin{itemize}
  \item 2-Jet VBF Tight \jra{} largest rate uncertainty \\ (45\% UEPS for GF)
  \item Gamma PDF has limitations in tool (intended for sidebands)
  \item Gamma can't be used in way we want for multiple signals
  \item Therefore, replace all uncertainties with one per signal
\end{itemize}
\begin{center}
\scriptsize
\begin{tabular}{ | p{4cm} | c | c | }
\hline
 & Expected Limit  & Observed Limit\\
 & ($\sigma/\sigma_\mathrm{SM}$) & ($\sigma/\sigma_\mathrm{SM}$) \\ \hline \hline
Paper & 11.7 & 8.2 \\ \hline
Replace all Rate Uncertainties with 45\% log-Normal for GF and 15\% log-Normal for VBF & 11.8 & 8.3 \\ \hline
Replace all Rate Uncertainties with 45\% Gamma for GF and 15\% Gamma for VBF & 11.9 & 8.3 \\ \hline
%Replace all Rate Uncertainties with 45\% log-Uniform for GF and 15\% log-Uniform for VBF & 13.8 & 9.7 \\ \hline
\end{tabular}
\\ \normalsize
\vspace{1ex}
\bf Gamma Changes Results Little over Log-Normal
\end{center}
\end{frame}

\begin{frame}
\frametitle{Limits for 0,1-Jet Tight BB 8 TeV: Gamma PDF}
\begin{itemize}
  \item Also check important 0,1-Jet Tight BB category
\end{itemize}
\begin{center}
\scriptsize
\begin{tabular}{ | p{5cm} | c | c | }
\hline
 & Expected Limit  & Observed Limit\\
 & ($\sigma/\sigma_\mathrm{SM}$) & ($\sigma/\sigma_\mathrm{SM}$) \\ \hline \hline
Paper & 16.1 & 27.9 \\ \hline
Replace all Rate Uncertainties with 20\% log-Normal for GF and 10\% log-Normal for VBF & 16.2 & 28.3 \\ \hline
Replace all Rate Uncertainties with 20\% Gamma for GF and 10\% Gamma for VBF & 16.3 & 28.8 \\ \hline
%Replace all Rate Uncertainties with 20\% log-Uniform for GF and 10\% log-Uniform for VBF & 18.2 & 31.1 \\ \hline
\end{tabular}
\\ \normalsize
\vspace{1ex}
\bf Gamma Changes Results Little over Log-Normal
\end{center}
\end{frame}

\begin{frame}
\frametitle{Limits: log-Uniform PDF}
\begin{itemize}
  \item log-Uniform PDF allows rates to float up and down by a factor of the uncertainty value e.g. 5 events $\pm$ 10\%
\end{itemize}
\begin{center}
\scriptsize
\begin{tabular}{ | l | p{4cm} | c | c | }
\hline
& & Expected Limit  & Observed Limit\\
Category & & ($\sigma/\sigma_\mathrm{SM}$) & ($\sigma/\sigma_\mathrm{SM}$) \\ \hline \hline
0,1-Jet Tight BB& 45\% log-Normal for GF and 15\% log-Normal for VBF & 11.8 & 8.3 \\ \hline
0,1-Jet Tight BB& 45\% log-Uniform for GF and 15\% log-Uniform for VBF & 13.8 & 9.7 \\ \hline
\hline
2-Jet Tight VBF& 20\% log-Normal for GF and 10\% log-Normal for VBF & 16.2 & 28.3 \\ \hline
2-Jet Tight VBF& 20\% log-Uniform for GF and 10\% log-Uniform for VBF & 18.2 & 31.1 \\ \hline
\end{tabular}
\\ \normalsize
\vspace{1ex}
\bf log-Uniform 10-15\% larger limits than log-Normal \\
Uncertainty \& signal strength able to move equally
\end{center}
\end{frame}

\begin{frame}
\frametitle{Conclusions}
\begin{itemize}
  \item Investigated influence of nuisance PDF choice by reviewer request
  \item Checked that nuisance parameters aren't pulled
  \item Compared Gamma v log-Normal PDF shapes--some difference for $\sim 50\%$ Uncertainty
  \item Gamma nuisance PDF limits negligibly different from default log-Normal
  \item log-Uniform nuisance PDF limits 10-15\% worse than log-Normal
  \begin{itemize}
    \item Expect limits to be worse, signal can be absorbed by nuisance instead of signal strength
  \end{itemize}
\end{itemize}
\begin{center}
\textbf{Would like to put in the reply to the reviewer that we tried an alternate Gamma PDF, and that limits didn't change, and also that the pulls are small}
\end{center}
\end{frame}

\begin{frame}
\begin{center}
\huge
\textbf{Backup}
\end{center}
\end{frame}

\begin{frame}
\frametitle{Technical Notes on Gamma Distribution}
\begin{equation}
  f(x,k,\theta) = \frac{1}{\Gamma(k)\theta^k} x^{k-1} \mathrm{e}^{-x/\theta}
\end{equation}
I choose to make the expected number of events the mean of the distribution:
\begin{equation}
\bar{x} = k\theta
\end{equation}
and the variance is:
\begin{equation}
\sigma_{x}^2 = k\theta^2
\end{equation}
The parameters are then set to:
\begin{equation}
k = (\text{Relative Error})^{-2}
\end{equation}
\begin{equation}
\theta =  \bar{x} (\text{Relative Error})^2
\end{equation}


\begin{center}
\textcolor{blue}{\tiny\underline{\url{http://en.wikipedia.org/wiki/Gamma\_distribution}}}
\end{center}

\end{frame}



\begin{frame}
\frametitle{Log-Uniform PDF Shape}
    \begin{center}
        \begin{tikzpicture}
            \node[anchor=south west,inner sep=0] at (0,0) {
                \includegraphics[height=60mm]{plots/nuisPdfShapesLogUnif.pdf}
            };
            %\draw[help lines,xstep=1,ystep=1] (0,0) grid (5.0,5.0);
            %\draw (2.3,4.2) node [fill=green!90!black,rounded corners,font=\scriptsize,text width=1.4cm,align=center] {For Twiki Approval};
        \end{tikzpicture}
        \\
        \vspace{1ex}
    \end{center}
\end{frame}

\end{document}
